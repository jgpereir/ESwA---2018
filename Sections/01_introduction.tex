\section{Introduction}

Crisis and emergency are ``sudden and usually unforeseen event that calls for immediate measures to minimise its adverse consequences'' \citep{dha1992internationally}. The term ``crisis communication'' is usually employed in two different contexts: when we refer to communication between organisations involved in managing the crisis and when we refer to necessity of the emergency management team to inform and alert the public about the emergency \citep{cdc2014}. The latter definition fits within the scope of our research, being developed considering to minimise the challenges faced by communicators while performing this task. 

During crisis and emergency situations, establishing a good communication between the emergency management team and the general public is a key step to minimise the impact on the affected public. However, the process to establish a good communication is not simple. Crisis and emergency are complex situations involving stress, panic, fear and uncertainty \citep{reynolds2007crisis} -- both to the communication team and the people affected directly or indirectly by the consequence of the crisis/emergency.

In addiction, the communicators need to ensure that the appropriate messages will be sent to each target audience according to their interests \citep{panamericanhealthorganization2009}. This includes writing the messages using appropriate vocabulary, providing the audience with only the most relevant information, ensure the consistency of each message and your trustworthiness. 

The current solutions for public communication during crisis and emergency situations were designed to disseminate one message to all affected public\citep{malizia2010sema4a}\citep{pereirachallenges}. In general, those solutions are focused almost totally entirely on the task of disseminating a public communication for several communications channels. The dissemination of messages to the affected public by several communications channel is an important step in the public communications of emergencies and crisis. On the other hand, it does not make sense to disseminate public communication messages by various communication channels if this message does not meet the information needs of each different target audiences. This is the main problem with this strategy of sending a single message for all target audience, the dissemination of information that is not in accordance with the information needs of the specific audiences could create an information overload in a situation that the public is incapable of juggling multiple facts \citep{cdc2014}. 

During the last few decades, many studies and guidelines of public communication of emergencies have been proposed in order to reduce emergency and service impacts  \citep{cdc2014}\citep{cisvGuide}\citep{panamericanhealthorganization2009}\citep{reynolds2007crisis}\citep{seeger2006best}\citep{tinker2010}. The problem on the creation and dissemination of public communication during crisis and emergency situations may be explained by the complexity and by the lives at risk involved. The need to inform affected people as quickly than possible and ensuring the consistence of the information in a situation that has ``more questions than answers \citep{cdc2014}'' is a great challenger to the emergency communicator.

 Effective communication with the public during emergencies will, in a simplified way, ,consistently provide reliable information, formulate consistent messages and respond to the interests of each public, and disseminate them through various communication channels in order to achieve the most amount of stakeholders quickly.

 In this work, we present a variability based model to help public communicators semi-automatically create and disseminate emergency public communications during emergency situations.  Our model was developed according to good practices present in the main guidelines of public communication of emergencies and by the knowledge developed in the RESCUER Project. 

The RESCUER Project aims at developing an interoperable solution to support command centres in quickly managing emergencies, based on reliable and intelligent analysis of crowdsourcing information mashed up with open data. This project was attended by experts in emergency management from two countries (Brazil and Austria) and in two scenarios of emergency: Industrial parks and Large Scale events.  

Our research also contributes with a study of the variability present in the process of public communication of emergencies. We map the commonalities in the different aspects relevant to the configuration of an solution for public communication of emergencies or that affect (directly or indirectly) the content of the public communications releases.

An important contribution of our work is our novel approach to generate emergency public communication from dynamic models who adapt to the current state of the emergency. In other words, content can be included or removed automatically according to the type of incident and/or target audience that is communicating, for example. In addition, we present four behaviours for varying the content of the sentences of the public communications according to the current status of the emergency.

Finally, another contribution of our work is our emergency communication solution called ``RESCUER News''. Developed as proof-of-concept of our proposed model, our solution was configured to work in three distinct scenarios of emergency in order to demonstrate the flexibility and configurability afforded by the variability based approach used in the development of our model. 


The remainder of this paper is structured as follows: Section \ref{sec:background} presents a background of concepts used during our research, Section \ref{sec:proposedModel} details our variability based model to semi-automatically generate and disseminate emergency public communications, Section \ref{sec:proofConcept} presents our proof-of-concept solution configured to three demonstrations that occurred in the context of the RESCUER Project; Section \ref{sec:relatedWork} presents a discussion on related works, and \ref{sec:conclusion} presents a summary of our findings and directions for future works.

%Establish an effective communication with the public during a crisis is a key measure for the crisis mitigation. A set of studies \cite{tinker2010}\cite{cdc2014}\cite{glik2007}\cite{seeger2006best} says that is essential maintaining the public informed about the occurrence of an emergency and its consequences. Among the benefits of this action, we can highlight: reduce crisis-related uncertainty; correct misunderstandings, rumours, or unclear facts; and reduce emotional turmoil and anxiety from the public. In addiction, this studies propose good communication principles among which we can highlight some basic principles: be first (for the public the first source is the most reliable); be right (accuracy establishes credibility); be credible (honesty and truthfulness are essential during an crisis); communicate repeatedly (whenever possible, keep the public informed); and communicate by different tools, media and communication channels (never trust on a single method of communication).