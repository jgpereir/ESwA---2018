\section{Background}\label{sec:background}

In this section, we briefly introduce main concepts used during the development of our work. 

\subsection{Public Communication of Emergencies}

Emergencies are defined as critical situations caused by incidents, natural or man-made, that require measures to be taken immediately to reduce their adverse consequences to life and property \citep{dha1992internationally}. The adverse and therefore undesired consequences of an emergency give rise to a crisis.
An emergency comprises a random and totally unexpected event. However, it presents patterns that can help communicators anticipate problems and be able to give an immediate response \citep{cdc2014}, hopefully avoiding a crisis.

However, emergency and crisis are used as synonym in the field of public communication. In this context, the term ``crisis communication'' is used with two meanings. The first one refers to the communication among organisations involved in the emergency management. The second one is related to the need of the emergency management to inform/alert the public about the emergency \citep{cdc2014}. The last definition is the one that best fits the scope of our research.

One key aspect of emergency management is communication, which can help people and organisations to handle the emergency situation in a better way. The challenge of this activity is to establish a good strategy for communication with the partners that should be informed.

Several studies have been conducted to propose good communication principles \citep{cdc2014} \citep{seeger2006best} \citep{goldfine2011best} \citep{glik2007} \citep{tinker2010} \citep{panamericanhealthorganization2009}. Some of the principles are essential for effective communication with the public during an emergency:

\begin{itemize}
   \item communicate repeatedly;
   \item be clear (use simple language and do not use technical terms, statistics or probabilities);
   \item communicate by different tools, media and communication channels (never trust on a single method of communication);
   \item transmit consistent information; and
   \item provide only relevant information.
   
 \end{itemize}
 
The studies \citep{cdc2014} and \citep{cisvGuide} also indicate the basic information the crisis/emergency communicators need to know about an emergency situation:
 
 \begin{itemize}
   \item WHAT happened?
   \item WHERE did it happen?
   \item WHEN did it happen?
   \item WHO is involved?
   \item HOW did it happen?
   \item WHAT is currently being done?
      
 \end{itemize}
 
 About the content of the public messages, \cite{reynolds2007crisis} propose a set of good characteristics that those messages need to have. This principal is called ``STARCC'' and it maintains that the messages must be:
 
 \begin{itemize}
   \item Simple: Frightened people do not want to hear big words;
   \item Timely: Frightened people want information NOW;
   \item Accurate: Frightened people will not get nuances, so give it straight;
   \item Relevant: Answer their questions and give action steps;
   \item Credible: Empathy and openness are key to credibility;
   \item Consistent: The slightest change in the message is upsetting and dissected by all.
      
 \end{itemize}

\textcolor{red}{coloque aqui uma frase pra fechar essa secao, e relacione ela com seu trabalho. Coloca tambem um link para a proxima secao. Levante a bola da necessidade de variabilidade, e ai fica justificado a secao seguinte}

%After analysing a set of studies in this field, we selected the Crisis and Emergency Risk Communication study \cite{cdc2014} to provide detailed information on crisis and emergency communication. CERC is

%A simple definition for Crisis and Emergency Risk Communication (CERC) is: a set of principles that aim to guide emergency managers to know what to say, when to say, how to tell and thus preserve or win the public's confidence during a crisis \cite{cdc2014}.

%CERC arises from the application of risk communication principles \cite{glik2007} in crisis communication \cite{cdc2014}, i.e., it incorporates communications about possible risk factors into the communication of the crisis evolution \cite{reynolds2005}.

\subsection{Feature Modeling}

Feature modeling is one of the Feature-Oriented Domains Models presented by \cite{Kang1990}. Feature-Oriented Domain Analysis (FODA) is a domain analysis method proposed to supports Software Reuse \citep{krueger1992software} at the functional an architectural levels. The main goal of this method is manage the commonalities and variabilities within a product line.  

In the Feature Model, the features are arranged in a hierarchical structure. One feature (parent feature) can have sub-features (child features). The relationship between this feature and his sub-features can be of the type AND (all child must be selected); Alternative (only one sub-feature can be selected), or OR (one or more can be selected). The relationship between features is defined by two type of constraints: Implies and Excludes. When a feature needs one or more feature we have a relationship of Implies. When is the opposite, the selection of one feature unable the selection of other feature, we have the relationship of Excludes \citep{batory2005feature}.

\textcolor{red}{coloque aqui uma frase pra fechar essa secao, e relacione ela com seu trabalho. Coloca tambem um link para a proxima secao, amarrando-a com public comunication e features...}