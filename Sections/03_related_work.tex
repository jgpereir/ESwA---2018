\section{Related Work}\label{sec:relatedWork}
The main contribution of our research is a variability based approach to support customised communication of the emergencies and its consequences targeted to proper audiences. Because of this, we explore the literature in order to identify related work in document variability.

There are proposals in the literature focusing exploring variability management techniques to build customised information for specific stakeholders. The information is usually represented by documents, as proposed by  \cite{penades2010}, which describes a process model for mass customisation of documents called Document Product Lines (DPL). Relying on SPL principles \citep{clements2002}, this work uses feature models \citep{Kang1990} to manage common and variable aspects of software development documents. The process model is realised by  DPLfw \citep{gomez2012dplfw}, a high-level design solution to support specialists on describing the variability of documents accordingly. DPLfw has been applied in several business domains to generate different DPLs (e.g. emergency plans \citep{gomez2012dplfw}, software manuals \citep{penades2012}, e-government \citep{penades2014},  customised recipes \citep{canos2013} and e-learning objects \citep{labib2015}). The DPLfw presents a complex process to specify and customise the document, this process is appropriated to large documents which demand large time intervals for your composition, not being proper to emergency public communication for not presenting such characteristics.

Karol et al. \citep{Karol2010} propose a tool for generating families of documents called Document Feature Mapper. This work also uses feature models to manage variability on families of documents. Each feature represents a fragment of the document (e.g. paragraphs, images) which varies as required. Despite the authors mapping the variability in several aspects, do not be considered the generating of documents in different formats, this is necessary for the dissemination of public communication using multiple communication channels. 

\cite{Rabiser2010} present an approach for automatic generation of product lines documents. They use the DOPLER \citep{rabiser2009} tool suite for modelling software artefacts and their respective variability and applies a Document Type Definition (DTD) called DOCBOOK \citep{walsh1999} for documents generation. Since this approach addresses the generation of product line documents, it cannot be applied to any other domain, such as public communication of emergencies.

%Limitações presentes em todos os papers
Public communication of emergencies requires the ability to build customised documents according to the audience in a quick way and deliver it considering all available communication infrastructures. The above mentioned proposals do not present solutions to cope with variability in communication network infrastructures. Furthermore, their tool support is limited, offering only plugins for the Eclipse\footnote{Eclipse - http://www.eclipse.org} Integrated Development Environment (IDE). Such an environment supports software development activities, it is not targeted to any other kinds of audience.

%como nosso trabalho atende às limitações dos outros artigos
Our proposal is designed to assist the public communication team. To do this, we design an assisted message edition of public communication models that vary, automatically, according to the emergency state. This allows the generation of specific messages according to the interest of each target groups from a unique public communication model.