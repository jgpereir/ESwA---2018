\begin{abstract}

A crisis is unpredictable by nature. Despite this, it presents patterns that can help communicators to anticipate problems to give faster and better responses to emergency situations. The current solutions for emergency public communication focus on the dissemination of a single message through different communication channels, for all audiences. This strategy is against the good communication studies because different audiences will receive information that is not of interest to them or that will not be presented in the best way for their understanding (e.g., technical and statistical information for the lay public). An inappropriate public communication message can amplify the perception of risk from at the public, and thus, be adding to the insecurity and lead to the creation of noises in communication that are difficult to fix in the progress of the crisis. Several studies have been conducted to propose good communication principles. Our research considers these principles to propose a computational model for public communication of emergencies. We mapped the variability in the process of communication of emergencies in order to ensure flexibility of configuration in different scenarios and types of emergencies. The main contribution of our research is a variability based model to support, semi-automatically, customised communication of the emergencies and its consequences targeted to proper audiences. In this approach, we defined four variability behaviours in the content of the sentences in order to support the adaptation according to the emergency state information. This research is being developed inside of a larger research project named RESCUER\footnote{Rescuer Project - http://www.rescuer-project.org} -- a joint Brazil-Europe initiative, involving nine research and industry organisations in four countries (Brazil, Austria, Germany and Spain).  As proof of concept, we develop the RESCUER News according to our conceptual model and demonstrate during three simulations cover distinct scenarios of emergency. 
\end{abstract}