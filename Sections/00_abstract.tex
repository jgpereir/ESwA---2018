\begin{abstract}

A crisis is unpredictable by nature, but it presents patterns that can help communicators to anticipate problems and give better responses to emergency situations. Current solutions for emergency public communication focus on the dissemination of a single message through different communication channels. This strategy goes against good communication practices because different audiences have different information needs. Inappropriate public communication messages creates noise and can amplify the perception of risk and insecurity.  Our research considers good communication principles to propose a computational model for public communication of emergencies. It maps and models variability in emergency communication processes to ensure flexibility of communication configuration in different emergency scenarios. This facilitates rapid construction of customised and consistent emergency communications, over different channels, for different audiences. The model was developed as part of a large research project named RESCUER, which uses crowd-sourcing to monitor and manage emergencies situations.  During the project, a proof of concept of the communication model, named RESCUER News, was build and validated in simulations involving real public and operational forces, over three distinct emergency scenarios. 
\end{abstract}